% Options for packages loaded elsewhere
\PassOptionsToPackage{unicode}{hyperref}
\PassOptionsToPackage{hyphens}{url}
%
\documentclass[
]{article}
\usepackage{amsmath,amssymb}
\usepackage{lmodern}
\usepackage{iftex}
\ifPDFTeX
  \usepackage[T1]{fontenc}
  \usepackage[utf8]{inputenc}
  \usepackage{textcomp} % provide euro and other symbols
\else % if luatex or xetex
  \usepackage{unicode-math}
  \defaultfontfeatures{Scale=MatchLowercase}
  \defaultfontfeatures[\rmfamily]{Ligatures=TeX,Scale=1}
\fi
% Use upquote if available, for straight quotes in verbatim environments
\IfFileExists{upquote.sty}{\usepackage{upquote}}{}
\IfFileExists{microtype.sty}{% use microtype if available
  \usepackage[]{microtype}
  \UseMicrotypeSet[protrusion]{basicmath} % disable protrusion for tt fonts
}{}
\makeatletter
\@ifundefined{KOMAClassName}{% if non-KOMA class
  \IfFileExists{parskip.sty}{%
    \usepackage{parskip}
  }{% else
    \setlength{\parindent}{0pt}
    \setlength{\parskip}{6pt plus 2pt minus 1pt}}
}{% if KOMA class
  \KOMAoptions{parskip=half}}
\makeatother
\usepackage{xcolor}
\usepackage[margin=1in]{geometry}
\usepackage{color}
\usepackage{fancyvrb}
\newcommand{\VerbBar}{|}
\newcommand{\VERB}{\Verb[commandchars=\\\{\}]}
\DefineVerbatimEnvironment{Highlighting}{Verbatim}{commandchars=\\\{\}}
% Add ',fontsize=\small' for more characters per line
\usepackage{framed}
\definecolor{shadecolor}{RGB}{248,248,248}
\newenvironment{Shaded}{\begin{snugshade}}{\end{snugshade}}
\newcommand{\AlertTok}[1]{\textcolor[rgb]{0.94,0.16,0.16}{#1}}
\newcommand{\AnnotationTok}[1]{\textcolor[rgb]{0.56,0.35,0.01}{\textbf{\textit{#1}}}}
\newcommand{\AttributeTok}[1]{\textcolor[rgb]{0.77,0.63,0.00}{#1}}
\newcommand{\BaseNTok}[1]{\textcolor[rgb]{0.00,0.00,0.81}{#1}}
\newcommand{\BuiltInTok}[1]{#1}
\newcommand{\CharTok}[1]{\textcolor[rgb]{0.31,0.60,0.02}{#1}}
\newcommand{\CommentTok}[1]{\textcolor[rgb]{0.56,0.35,0.01}{\textit{#1}}}
\newcommand{\CommentVarTok}[1]{\textcolor[rgb]{0.56,0.35,0.01}{\textbf{\textit{#1}}}}
\newcommand{\ConstantTok}[1]{\textcolor[rgb]{0.00,0.00,0.00}{#1}}
\newcommand{\ControlFlowTok}[1]{\textcolor[rgb]{0.13,0.29,0.53}{\textbf{#1}}}
\newcommand{\DataTypeTok}[1]{\textcolor[rgb]{0.13,0.29,0.53}{#1}}
\newcommand{\DecValTok}[1]{\textcolor[rgb]{0.00,0.00,0.81}{#1}}
\newcommand{\DocumentationTok}[1]{\textcolor[rgb]{0.56,0.35,0.01}{\textbf{\textit{#1}}}}
\newcommand{\ErrorTok}[1]{\textcolor[rgb]{0.64,0.00,0.00}{\textbf{#1}}}
\newcommand{\ExtensionTok}[1]{#1}
\newcommand{\FloatTok}[1]{\textcolor[rgb]{0.00,0.00,0.81}{#1}}
\newcommand{\FunctionTok}[1]{\textcolor[rgb]{0.00,0.00,0.00}{#1}}
\newcommand{\ImportTok}[1]{#1}
\newcommand{\InformationTok}[1]{\textcolor[rgb]{0.56,0.35,0.01}{\textbf{\textit{#1}}}}
\newcommand{\KeywordTok}[1]{\textcolor[rgb]{0.13,0.29,0.53}{\textbf{#1}}}
\newcommand{\NormalTok}[1]{#1}
\newcommand{\OperatorTok}[1]{\textcolor[rgb]{0.81,0.36,0.00}{\textbf{#1}}}
\newcommand{\OtherTok}[1]{\textcolor[rgb]{0.56,0.35,0.01}{#1}}
\newcommand{\PreprocessorTok}[1]{\textcolor[rgb]{0.56,0.35,0.01}{\textit{#1}}}
\newcommand{\RegionMarkerTok}[1]{#1}
\newcommand{\SpecialCharTok}[1]{\textcolor[rgb]{0.00,0.00,0.00}{#1}}
\newcommand{\SpecialStringTok}[1]{\textcolor[rgb]{0.31,0.60,0.02}{#1}}
\newcommand{\StringTok}[1]{\textcolor[rgb]{0.31,0.60,0.02}{#1}}
\newcommand{\VariableTok}[1]{\textcolor[rgb]{0.00,0.00,0.00}{#1}}
\newcommand{\VerbatimStringTok}[1]{\textcolor[rgb]{0.31,0.60,0.02}{#1}}
\newcommand{\WarningTok}[1]{\textcolor[rgb]{0.56,0.35,0.01}{\textbf{\textit{#1}}}}
\usepackage{graphicx}
\makeatletter
\def\maxwidth{\ifdim\Gin@nat@width>\linewidth\linewidth\else\Gin@nat@width\fi}
\def\maxheight{\ifdim\Gin@nat@height>\textheight\textheight\else\Gin@nat@height\fi}
\makeatother
% Scale images if necessary, so that they will not overflow the page
% margins by default, and it is still possible to overwrite the defaults
% using explicit options in \includegraphics[width, height, ...]{}
\setkeys{Gin}{width=\maxwidth,height=\maxheight,keepaspectratio}
% Set default figure placement to htbp
\makeatletter
\def\fps@figure{htbp}
\makeatother
\setlength{\emergencystretch}{3em} % prevent overfull lines
\providecommand{\tightlist}{%
  \setlength{\itemsep}{0pt}\setlength{\parskip}{0pt}}
\setcounter{secnumdepth}{-\maxdimen} % remove section numbering
\ifLuaTeX
  \usepackage{selnolig}  % disable illegal ligatures
\fi
\IfFileExists{bookmark.sty}{\usepackage{bookmark}}{\usepackage{hyperref}}
\IfFileExists{xurl.sty}{\usepackage{xurl}}{} % add URL line breaks if available
\urlstyle{same} % disable monospaced font for URLs
\hypersetup{
  pdftitle={Lab 1 Probability Distributions},
  pdfauthor={Student Name},
  hidelinks,
  pdfcreator={LaTeX via pandoc}}

\title{Lab 1 Probability Distributions}
\author{Student Name}
\date{16/1/2023}

\begin{document}
\maketitle

{
\setcounter{tocdepth}{2}
\tableofcontents
}
\hypertarget{lab-1-lab-manual-exercise}{%
\section{Lab 1 Lab Manual Exercise}\label{lab-1-lab-manual-exercise}}

copy and paste your work by following each example from the lab manual
for this exercise

\begin{Shaded}
\begin{Highlighting}[]
\FunctionTok{rm}\NormalTok{(}\AttributeTok{list =} \FunctionTok{setdiff}\NormalTok{(}\FunctionTok{ls}\NormalTok{(), }\FunctionTok{lsf.str}\NormalTok{()))}

\CommentTok{\# Plot Normal Distributions with }
\CommentTok{\#{-}{-}{-}{-}{-}{-}{-}{-}{-}{-}{-}{-}{-}{-}{-}{-}{-}{-}{-}{-}{-}{-}{-}{-}{-}{-}{-}{-}{-}{-}{-}{-}{-}{-}{-}{-}{-}{-}{-}{-}{-}}
\CommentTok{\# Same standard deviation, different mean}
\CommentTok{\#{-}{-}{-}{-}{-}{-}{-}{-}{-}{-}{-}{-}{-}{-}{-}{-}{-}{-}{-}{-}{-}{-}{-}{-}{-}{-}{-}{-}{-}{-}{-}{-}{-}{-}{-}{-}{-}{-}{-}{-}{-}}
\CommentTok{\# Mean 1, sd 1}
\CommentTok{\# Grid of X{-}axis values}
\NormalTok{x }\OtherTok{\textless{}{-}} \FunctionTok{seq}\NormalTok{(}\SpecialCharTok{{-}}\DecValTok{4}\NormalTok{, }\DecValTok{10}\NormalTok{, }\FloatTok{0.1}\NormalTok{)}

\FunctionTok{plot}\NormalTok{(x, }\FunctionTok{dnorm}\NormalTok{(x, }\AttributeTok{mean =} \DecValTok{1}\NormalTok{, }\AttributeTok{sd =} \FloatTok{0.1}\NormalTok{), }\AttributeTok{type =} \StringTok{"l"}\NormalTok{,}
     \AttributeTok{ylim =} \FunctionTok{c}\NormalTok{(}\DecValTok{0}\NormalTok{, }\DecValTok{5}\NormalTok{), }\AttributeTok{ylab =} \StringTok{""}\NormalTok{, }\AttributeTok{lwd =} \DecValTok{2}\NormalTok{, }\AttributeTok{col =} \StringTok{"red"}\NormalTok{)}
\CommentTok{\# Mean 3, sd 1}
\FunctionTok{lines}\NormalTok{(x, }\FunctionTok{dnorm}\NormalTok{(x, }\AttributeTok{mean =} \DecValTok{3}\NormalTok{, }\AttributeTok{sd =} \DecValTok{1}\NormalTok{), }\AttributeTok{col =} \StringTok{"blue"}\NormalTok{, }\AttributeTok{lty =} \DecValTok{1}\NormalTok{, }\AttributeTok{lwd =} \DecValTok{2}\NormalTok{)}
\end{Highlighting}
\end{Shaded}

\includegraphics{lab_16thjan_files/figure-latex/unnamed-chunk-1-1.pdf}

\begin{Shaded}
\begin{Highlighting}[]
\CommentTok{\# \# Function Syntax}
\CommentTok{\# }
\CommentTok{\# function\_name \textless{}{-} function(arg\_1, arg\_2, ...) \{}
\CommentTok{\#    Function body }
\CommentTok{\# \}}
\end{Highlighting}
\end{Shaded}

\begin{Shaded}
\begin{Highlighting}[]
\CommentTok{\# Calculate the 60th \%ile of the standard normal.}
\FunctionTok{qnorm}\NormalTok{(}\FloatTok{0.6}\NormalTok{,}\DecValTok{0}\NormalTok{,}\DecValTok{1}\NormalTok{)}
\end{Highlighting}
\end{Shaded}

\begin{verbatim}
## [1] 0.2533471
\end{verbatim}

\begin{Shaded}
\begin{Highlighting}[]
\CommentTok{\# Calculate the probability that a value lies below 0.8 in the standard normal distribution}
\FunctionTok{pnorm}\NormalTok{(}\FloatTok{0.8}\NormalTok{,}\DecValTok{0}\NormalTok{,}\DecValTok{1}\NormalTok{)}
\end{Highlighting}
\end{Shaded}

\begin{verbatim}
## [1] 0.7881446
\end{verbatim}

\begin{Shaded}
\begin{Highlighting}[]
\CommentTok{\# Draw 1000 samples of 30 coin tosses with p(heads) = 0.6 \# and plot the distribution}
\CommentTok{\# Syntax: rbinom (\# observations, \# trials per observation, probability of success )}
\FunctionTok{hist}\NormalTok{(}\FunctionTok{rbinom}\NormalTok{(}\DecValTok{1000}\NormalTok{,}\DecValTok{30}\NormalTok{,}\FloatTok{0.6}\NormalTok{))}
\end{Highlighting}
\end{Shaded}

\includegraphics{lab_16thjan_files/figure-latex/unnamed-chunk-3-1.pdf}

\begin{Shaded}
\begin{Highlighting}[]
\CommentTok{\# Do the above with only 10 trials per observation}
\FunctionTok{hist}\NormalTok{(}\FunctionTok{rbinom}\NormalTok{(}\DecValTok{1000}\NormalTok{,}\DecValTok{10}\NormalTok{,}\FloatTok{0.6}\NormalTok{))}
\end{Highlighting}
\end{Shaded}

\includegraphics{lab_16thjan_files/figure-latex/unnamed-chunk-3-2.pdf}

\begin{Shaded}
\begin{Highlighting}[]
\CommentTok{\# Do the above with 100 observations and 5 trials per observation}
\FunctionTok{hist}\NormalTok{(}\FunctionTok{rbinom}\NormalTok{(}\DecValTok{100}\NormalTok{,}\DecValTok{5}\NormalTok{,}\FloatTok{0.6}\NormalTok{))}
\end{Highlighting}
\end{Shaded}

\includegraphics{lab_16thjan_files/figure-latex/unnamed-chunk-3-3.pdf}

\begin{Shaded}
\begin{Highlighting}[]
\CommentTok{\# Transformations between probability distributions}

\CommentTok{\# generate 1000 trials from a normal distribution}
\NormalTok{normal.a }\OtherTok{\textless{}{-}} \FunctionTok{rnorm}\NormalTok{( }\AttributeTok{n=}\DecValTok{1000}\NormalTok{, }\AttributeTok{mean=}\DecValTok{0}\NormalTok{, }\AttributeTok{sd=}\DecValTok{1}\NormalTok{ ) }
\FunctionTok{hist}\NormalTok{( normal.a )}
\end{Highlighting}
\end{Shaded}

\includegraphics{lab_16thjan_files/figure-latex/unnamed-chunk-4-1.pdf}

\begin{Shaded}
\begin{Highlighting}[]
\CommentTok{\#next, we generate a chi{-}square distribution with 3 \#degrees of freedom:}

\NormalTok{normal.b }\OtherTok{\textless{}{-}} \FunctionTok{rnorm}\NormalTok{( }\AttributeTok{n=}\DecValTok{1000}\NormalTok{ )  }\CommentTok{\# another set of normally distributed data}
\NormalTok{normal.c }\OtherTok{\textless{}{-}} \FunctionTok{rnorm}\NormalTok{( }\AttributeTok{n=}\DecValTok{1000}\NormalTok{ )  }\CommentTok{\# and another!}

\CommentTok{\# Take the SUM of SQUARES of the above 3 normally distributed variables a, b, and c}
\NormalTok{chi.sq}\FloatTok{.3} \OtherTok{\textless{}{-}}\NormalTok{ (normal.a)}\SpecialCharTok{\^{}}\DecValTok{2} \SpecialCharTok{+}\NormalTok{ (normal.b)}\SpecialCharTok{\^{}}\DecValTok{2} \SpecialCharTok{+}\NormalTok{ (normal.c)}\SpecialCharTok{\^{}}\DecValTok{2} 

\CommentTok{\# and the resulting chi.sq.3 variable should contain 1000 observations that follow a chi{-}square distribution with 3 degrees of freedom. You can use the hist() function to have a look at these observations yourself}

\FunctionTok{hist}\NormalTok{(chi.sq}\FloatTok{.3}\NormalTok{)}
\end{Highlighting}
\end{Shaded}

\includegraphics{lab_16thjan_files/figure-latex/unnamed-chunk-4-2.pdf}

\begin{Shaded}
\begin{Highlighting}[]
\DocumentationTok{\#\# Now how do we get to a t{-}distribution from Normal and chi{-}sq distributions?}
\CommentTok{\# First, take a scaled chi{-}sq by dividing it by the degrees of freedom}
\NormalTok{scaled.chi.sq}\FloatTok{.3} \OtherTok{\textless{}{-}}\NormalTok{ chi.sq}\FloatTok{.3} \SpecialCharTok{/} \DecValTok{3}
\CommentTok{\# Then take a normally distributed variable and divide them by the square root of the scaled chi{-}sq variable to get a t{-}distribution with the same degrees of freedom}

\NormalTok{normal.d }\OtherTok{\textless{}{-}} \FunctionTok{rnorm}\NormalTok{( }\AttributeTok{n=}\DecValTok{1000}\NormalTok{ )                }\CommentTok{\# yet another \#set of normally distributed data}
\NormalTok{t}\FloatTok{.3} \OtherTok{\textless{}{-}}\NormalTok{ normal.d }\SpecialCharTok{/} \FunctionTok{sqrt}\NormalTok{( scaled.chi.sq}\FloatTok{.3}\NormalTok{ )  }\CommentTok{\# divide by \#square root of scaled chi{-}square to get t}
\FunctionTok{hist}\NormalTok{ (t}\FloatTok{.3}\NormalTok{)}
\end{Highlighting}
\end{Shaded}

\includegraphics{lab_16thjan_files/figure-latex/unnamed-chunk-4-3.pdf}

\begin{Shaded}
\begin{Highlighting}[]
\DocumentationTok{\#\# To get to an F distribution, take the ratio between two scaled chi{-}sq distributions.}
\CommentTok{\# F distribution with 3 and 20 degrees of freedom:}
\CommentTok{\# first take two chi{-}sq variables, with 3 dof and 20 dof respectively, and take the ratio:}

\NormalTok{chi.sq}\FloatTok{.20} \OtherTok{\textless{}{-}} \FunctionTok{rchisq}\NormalTok{( }\DecValTok{1000}\NormalTok{, }\DecValTok{20}\NormalTok{)                 }\CommentTok{\# generate chi square data with df = 20...}
\NormalTok{scaled.chi.sq}\FloatTok{.20} \OtherTok{\textless{}{-}}\NormalTok{ chi.sq}\FloatTok{.20} \SpecialCharTok{/} \DecValTok{20}             \CommentTok{\# scale \#the chi square variable...}
\NormalTok{F.}\FloatTok{3.20} \OtherTok{\textless{}{-}}\NormalTok{  scaled.chi.sq}\FloatTok{.3}  \SpecialCharTok{/}\NormalTok{ scaled.chi.sq}\FloatTok{.20} \CommentTok{\# take the ratio of the two chi squares...}
\FunctionTok{hist}\NormalTok{( F.}\FloatTok{3.20}\NormalTok{, }\AttributeTok{freq =} \ConstantTok{FALSE}\NormalTok{)                                 }\CommentTok{\# ... and draw a picture}
\FunctionTok{curve}\NormalTok{(}\FunctionTok{df}\NormalTok{(x, }\DecValTok{3}\NormalTok{, }\DecValTok{20}\NormalTok{), }
      \AttributeTok{col=}\StringTok{"darkblue"}\NormalTok{, }\AttributeTok{lwd=}\DecValTok{2}\NormalTok{, }\AttributeTok{add=}\ConstantTok{TRUE}\NormalTok{, }\AttributeTok{yaxt=}\StringTok{"n"}\NormalTok{)}
\end{Highlighting}
\end{Shaded}

\includegraphics{lab_16thjan_files/figure-latex/unnamed-chunk-4-4.pdf}

\begin{Shaded}
\begin{Highlighting}[]
\DocumentationTok{\#\# The curve above confirms this looks similar if you use the R built{-}in function df (just like dnorm, but for the F distribution)}
\end{Highlighting}
\end{Shaded}

\hypertarget{lab-1-generalization-exercises}{%
\section{Lab 1 Generalization
exercises}\label{lab-1-generalization-exercises}}

use the code from above to attempt to solve the extra things we ask you
do for this assignment

\begin{Shaded}
\begin{Highlighting}[]
\CommentTok{\# Q1 Plot a normal distribution with mean = 2, s.d. = 0.4}


\CommentTok{\# Q2 Calculate the 85th \%ile of the above distribution.}

\CommentTok{\# Q3 Calculate the probability that a value lies in between 1 and 2 given the above distribution}

\CommentTok{\# Q4 Plot a simulated t{-}distribution with 5 degrees of freedom.}
\end{Highlighting}
\end{Shaded}

\hypertarget{lab-1-written-answer-question}{%
\section{Lab 1 Written answer
question}\label{lab-1-written-answer-question}}

Write your answer here.

\end{document}
